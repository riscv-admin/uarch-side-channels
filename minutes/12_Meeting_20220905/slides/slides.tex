\documentclass[11pt]{beamer}
%\usetheme{Singapore}
\usepackage[utf8]{inputenc}
\usepackage[english]{babel}
\usepackage{amsmath}
\usepackage{amsfonts}
\usepackage{amssymb}
\usepackage{graphicx}
%\author{}
%\title{}
%\setbeamercovered{transparent} 
\setbeamertemplate{navigation symbols}{} 
%\logo{} 
%\institute{} 
%\date{} 
%\subject{} 
\begin{document}


\begin{frame}[allowframebreaks]{Charter proposition}
 
        \begin{small}
Timing covert channels are used to exfiltrate confidential data using microarchitectural states as a medium for communications. These channels are particularly relevant in the context of microarchitectural attacks such as Spectre and Meltdown.

The Microarchitecture Side-Channel Resistant Instruction Spans Task Group (proposed short name: uSCR-IS TG) will define a small ISA extension to prevent malicious covert channels. More precisely, we will introduce a notion of side-channel resistant instruction spans, such that covert channels can be prevented across instruction spans by adapting the microarchitecture. Introducing instruction spans as an architectural feature makes it possible for higher-level program logic to declare that a sequence of instructions should be microarchitecturally isolated within a larger instruction stream (for example, a span of instructions that implement a cryptographic operation may be isolated to protect against side-channel attacks). The proposed RISC-V uSCR-IS TG will collaborate to produce:

\begin{enumerate}
\item A small ISA extension (possibly no more than one or two instructions, or only a new CSR).
\item A security guide: defining threat models, developing rationale, etc. -> intended for security engineers.
\item An implementation guide, focusing on the principles of microarchitecture design that enable protection against covert channels. -> intended for hardware engineers.
\item A proof-of-concept implementation, including both a prototype RISC-V core and compiler managing the necessary intrinsics.
\item A test strategy guide, including a test suite for common covert channels.
\end{enumerate}
The TG will work with the appropriate Priv/Unpriv ISA committee, Architecture Review Committee, and Security HC to determine which parts of the work should follow the standard ISA specification process, Fast Track process, or non-ISA process, and how other recent policy or process changes may apply (such as the discussion around the use of hint instructions in CFI).

        \end{small}

\end{frame}


\end{document}