\documentclass[11pt]{beamer}
%\usetheme{Singapore}
\usepackage[utf8]{inputenc}
\usepackage[english]{babel}
\usepackage{amsmath}
\usepackage{amsfonts}
\usepackage{amssymb}
\usepackage{graphicx}
\usepackage{minted}
\usepackage{pifont}% http://ctan.org/pkg/pifont
\newcommand{\cmark}{\textcolor{green}{\ding{51}}}%
\newcommand{\xmark}{\textcolor{red}{\ding{55}}}%
%\author{}
%\title{}
%\setbeamercovered{transparent} 
\setbeamertemplate{navigation symbols}{} 
%\logo{} 
%\institute{} 
%\date{} 
%\subject{} 
\begin{document}

\section*{Microarchitecture Side-Channel Resistant Instructions Spans (scrispans)}
\frame{\sectionpage}

\begin{frame}{Examples semantics}
 
        \begin{itemize}
                \item \texttt{spansec.create [policies]}: create a new scrispan with given optional set of security policies (flags).
                \item \texttt{spansec.save rd}: save the current scrispan configuration (ID + security policy) in a register.
                \item \texttt{spansec.restore rs1}: restore a scrispan configuration from a register holding ID and security policy.
        \end{itemize}
                \textbf{We assume that a change of scrispan ID implies microarchitectural isolation.}
        \begin{itemize}
                \item \texttt{spansec.alter [policies]}: in addition we could add an instruction that keeps the ID unchanged but modify the security policies.
        \end{itemize}


        \textbf{We are not discussing the implementation for now. Have you alternative semantics in mind ?}
\end{frame}

\begin{frame}{Security policies}
        \texttt{[policies]} is the set of security policies applied to a scrispan.

        \begin{itemize}
                \item \texttt{no speculation}
                \item \texttt{inline memory encryption}
                \item \texttt{constant time execution}
        \end{itemize}

        \textit{Personal comment (Ronan): security policies add complexity, but not sure if useful. Suggested policies could be better achieved with other means.}\\
        \textbf{Do you have strong use cases for security policies ?}
\end{frame}

\begin{frame}[fragile]{Usecase \#1: web server}
        \begin{block}{Goal: isolate users}
                Each process has its own scrispan.
                \begin{minted}{c}
// creating process P1
spansec.create

// during switch from P1 to P2
spansec.save P1
spansec.restore P2   
                \end{minted}
                
        \end{block}

        \begin{alertblock}{Comments}
                \begin{itemize}
                        \item \cmark{} This explicit span gymnastic ensures proper microarchitectural state isolation.
                        \item \xmark{} In this precise use case, we could tie the scrispan to the ASID (Address Space ID) instead.
                \end{itemize}
        \end{alertblock}
        
\end{frame}

\begin{frame}[fragile]{Pattern: Microarchitectural state poisoning}
        \begin{block}{Goal: isolating user poisoned uarch state from data processing}
                \begin{minted}{c}
// span A: user controlled, span B: data processing
spansec.restore A
state_poisoning();
spansec.save A

spansec.restore B
data_processing();
spansec.save B
                \end{minted}
        \end{block}

        \begin{alertblock}{Comments}
                Since poisoning can be done in the same or another address space (cf \url{https://transient.fail}), we cannot rely on tying the scrispan with the ASID.
                
        \end{alertblock}
\end{frame}

\begin{frame}[fragile,allowframebreaks]{Simultaneous multithreading (SMT)}
        Example from \url{https://github.com/IAIK/transientfail/blob/master/pocs/spectre/RSB/sa_ip/main.c}, two software threads in the same address space.\\
\begin{block}{Pattern}
\begin{minted}{c}
pthread_t attacker_thread;
pthread_t victim_thread;
pthread_create(&attacker_thread, 0, attacker, 0);
pthread_create(&victim_thread, 0, victim, 0);
\end{minted}
\end{block}

\begin{block}{Solution}
\begin{minted}{c}
// pthread_create should start with
spansec.create
\end{minted}
\end{block}
\framebreak
\begin{alertblock}{Hardware consequences}
        \begin{itemize}
                \item They are scheduled on different cores.
                \item If the hardware allows, the harts are isolated in the same core.
                \item They are not executed in parallel but sequentially, with proper microarchitectural flushing when switching between threads.
        \end{itemize}
\end{alertblock}
\end{frame}

\begin{frame}[fragile,allowframebreaks]{Spectre-PHT (v1)}
        \begin{block}{Vulnerabiliy}
                \begin{minted}{c}
if (x < array1_size) {
    y = array2[array1[x] * 4096];
}
//speculative load 1: array1[x] -> accessing secret
//speculative load 2: array2[...] -> leaking gadget
                \end{minted}
        \end{block}        

        \begin{block}{Solution 1: disabling speculation}
                \begin{minted}{c}
spansec.alter +nospeculation
if (x < array1_size) {
    y = array2[array1[x] * 4096];
}
spansec.alter -nospeculation
                \end{minted}
        \end{block}
\framebreak
        \begin{block}{Solution 2: isolating poisoned state}
                \begin{minted}{c}
if (x < array1_size) {
    spansec.save A
    spansec.create
    y = array2[array1[x] * 4096];
    spansec.restore A
}
                \end{minted}
        \end{block}

        \begin{alertblock}{Comments}
                \begin{itemize}
                        \item \xmark{} \texttt{nospeculation}: poor portability, precise behaviour is very hardware dependent, overkill ? (How to allow speculation across scrispan ?).
                        \item \xmark{} \texttt{state isolation}: very bad performances.
                        \item Probably needs another mechanism.
                \end{itemize}
        \end{alertblock}
\end{frame}
\end{document}